% Options for packages loaded elsewhere
\PassOptionsToPackage{unicode}{hyperref}
\PassOptionsToPackage{hyphens}{url}
\PassOptionsToPackage{dvipsnames,svgnames,x11names}{xcolor}
%
\documentclass[
  letterpaper,
  DIV=11,
  numbers=noendperiod]{scrartcl}

\usepackage{amsmath,amssymb}
\usepackage{iftex}
\ifPDFTeX
  \usepackage[T1]{fontenc}
  \usepackage[utf8]{inputenc}
  \usepackage{textcomp} % provide euro and other symbols
\else % if luatex or xetex
  \usepackage{unicode-math}
  \defaultfontfeatures{Scale=MatchLowercase}
  \defaultfontfeatures[\rmfamily]{Ligatures=TeX,Scale=1}
\fi
\usepackage{lmodern}
\ifPDFTeX\else  
    % xetex/luatex font selection
\fi
% Use upquote if available, for straight quotes in verbatim environments
\IfFileExists{upquote.sty}{\usepackage{upquote}}{}
\IfFileExists{microtype.sty}{% use microtype if available
  \usepackage[]{microtype}
  \UseMicrotypeSet[protrusion]{basicmath} % disable protrusion for tt fonts
}{}
\makeatletter
\@ifundefined{KOMAClassName}{% if non-KOMA class
  \IfFileExists{parskip.sty}{%
    \usepackage{parskip}
  }{% else
    \setlength{\parindent}{0pt}
    \setlength{\parskip}{6pt plus 2pt minus 1pt}}
}{% if KOMA class
  \KOMAoptions{parskip=half}}
\makeatother
\usepackage{xcolor}
\setlength{\emergencystretch}{3em} % prevent overfull lines
\setcounter{secnumdepth}{-\maxdimen} % remove section numbering
% Make \paragraph and \subparagraph free-standing
\ifx\paragraph\undefined\else
  \let\oldparagraph\paragraph
  \renewcommand{\paragraph}[1]{\oldparagraph{#1}\mbox{}}
\fi
\ifx\subparagraph\undefined\else
  \let\oldsubparagraph\subparagraph
  \renewcommand{\subparagraph}[1]{\oldsubparagraph{#1}\mbox{}}
\fi


\providecommand{\tightlist}{%
  \setlength{\itemsep}{0pt}\setlength{\parskip}{0pt}}\usepackage{longtable,booktabs,array}
\usepackage{calc} % for calculating minipage widths
% Correct order of tables after \paragraph or \subparagraph
\usepackage{etoolbox}
\makeatletter
\patchcmd\longtable{\par}{\if@noskipsec\mbox{}\fi\par}{}{}
\makeatother
% Allow footnotes in longtable head/foot
\IfFileExists{footnotehyper.sty}{\usepackage{footnotehyper}}{\usepackage{footnote}}
\makesavenoteenv{longtable}
\usepackage{graphicx}
\makeatletter
\def\maxwidth{\ifdim\Gin@nat@width>\linewidth\linewidth\else\Gin@nat@width\fi}
\def\maxheight{\ifdim\Gin@nat@height>\textheight\textheight\else\Gin@nat@height\fi}
\makeatother
% Scale images if necessary, so that they will not overflow the page
% margins by default, and it is still possible to overwrite the defaults
% using explicit options in \includegraphics[width, height, ...]{}
\setkeys{Gin}{width=\maxwidth,height=\maxheight,keepaspectratio}
% Set default figure placement to htbp
\makeatletter
\def\fps@figure{htbp}
\makeatother

\KOMAoption{captions}{tableheading}
\makeatletter
\makeatother
\makeatletter
\makeatother
\makeatletter
\@ifpackageloaded{caption}{}{\usepackage{caption}}
\AtBeginDocument{%
\ifdefined\contentsname
  \renewcommand*\contentsname{Table of contents}
\else
  \newcommand\contentsname{Table of contents}
\fi
\ifdefined\listfigurename
  \renewcommand*\listfigurename{List of Figures}
\else
  \newcommand\listfigurename{List of Figures}
\fi
\ifdefined\listtablename
  \renewcommand*\listtablename{List of Tables}
\else
  \newcommand\listtablename{List of Tables}
\fi
\ifdefined\figurename
  \renewcommand*\figurename{Figure}
\else
  \newcommand\figurename{Figure}
\fi
\ifdefined\tablename
  \renewcommand*\tablename{Table}
\else
  \newcommand\tablename{Table}
\fi
}
\@ifpackageloaded{float}{}{\usepackage{float}}
\floatstyle{ruled}
\@ifundefined{c@chapter}{\newfloat{codelisting}{h}{lop}}{\newfloat{codelisting}{h}{lop}[chapter]}
\floatname{codelisting}{Listing}
\newcommand*\listoflistings{\listof{codelisting}{List of Listings}}
\makeatother
\makeatletter
\@ifpackageloaded{caption}{}{\usepackage{caption}}
\@ifpackageloaded{subcaption}{}{\usepackage{subcaption}}
\makeatother
\makeatletter
\@ifpackageloaded{tcolorbox}{}{\usepackage[skins,breakable]{tcolorbox}}
\makeatother
\makeatletter
\@ifundefined{shadecolor}{\definecolor{shadecolor}{rgb}{.97, .97, .97}}
\makeatother
\makeatletter
\makeatother
\makeatletter
\makeatother
\ifLuaTeX
  \usepackage{selnolig}  % disable illegal ligatures
\fi
\IfFileExists{bookmark.sty}{\usepackage{bookmark}}{\usepackage{hyperref}}
\IfFileExists{xurl.sty}{\usepackage{xurl}}{} % add URL line breaks if available
\urlstyle{same} % disable monospaced font for URLs
\hypersetup{
  pdftitle={Hito 1},
  pdfauthor={Juan Carlos Díaz},
  colorlinks=true,
  linkcolor={blue},
  filecolor={Maroon},
  citecolor={Blue},
  urlcolor={Blue},
  pdfcreator={LaTeX via pandoc}}

\title{Hito 1}
\author{Juan Carlos Díaz}
\date{}

\begin{document}
\maketitle
\ifdefined\Shaded\renewenvironment{Shaded}{\begin{tcolorbox}[interior hidden, boxrule=0pt, borderline west={3pt}{0pt}{shadecolor}, sharp corners, enhanced, breakable, frame hidden]}{\end{tcolorbox}}\fi

\hypertarget{descripciuxf3n-activo-y-contexto-hito-1}{%
\subsection{Descripción Activo y Contexto (Hito
1)}\label{descripciuxf3n-activo-y-contexto-hito-1}}

Fecha de entrega: Jueves 28 de Septiembre 23:59.

\hypertarget{definiciuxf3n}{%
\subsubsection{Definición}\label{definiciuxf3n}}

En el contexto mundial, siempre se ha visto la industria bancaria como
de las más grandes del mundo y así es pues, tienen la enorme
responsabilidad de custodiar los dineros de las personas y prestar para
obtener rentabilidad.

Como se menciona en algunas páginas web, un banco es una institución
financiera, que acepta depósitos y crea depósitos, lo que se conoce como
cuentas bancarias, además se ofrecen servicios financieros de créditos y
préstamos.

La importancia de los bancos, como dice el paper: ``Business culture and
dishonesty in the banking industry'' (Cohn et al., 2014) juegan un rol
muy importante en la economía, siendo los pilares de esta y sus
servicios son claves para el desarrollo económico, por lo tanto, deben
estar regulados y es lo que pasa en la mayoría de los países, los bancos
están muy regulados.

Para el caso del proyecto se decidió investigar sobre el Banco de Chile,
una institución financiara que proporciona productos y servicios
financieros a empresas y personas. Es una sociedad anónima cerrada y
posee cerca de US\$50 mil millones de activos. Ofrece productos de
cuentas corrientes, de ahorro, créditos, tarjetas de crédito y débito,
fondos mutuos, inversiones y servicios de banca. Su participación en el
mercado es del 25\%, con 1,9 millones de clientes y un grado de
inversión S\&P: A+; Moody's:Aa3 (\emph{SANTIAGOX}, n.d.)

\hypertarget{motivaciuxf3n}{%
\subsubsection{Motivación}\label{motivaciuxf3n}}

En el contexto nacional, cuando se habla de la industria bancaria se
suele pensar en corrupción o que el dinero del país esta en manos de
unos pocos. Eso motiva la investigación de este trabajo, complementando
esto con el paper mencionado anteriormente en general la industria
bancaria es honesta, pero por algunos hechos se ha ensuciado y se busca
conocer mejor su funcionamiento para entenderlo mejor. (Cohn et al.,
2014)

En Chile hay cerca de 24 bancos establecidos, pero el que tiene un mayor
patrimonio y el elegido a desarrollar es el Banco de Chile. Lo que
motiva a invertir en un banco es más que nada tener la seguridad de que
el dinero invertido no se vaya a perder o que el banco no vaya a
quebrar, en ese sentido lo que motiva a invertir en el Banco de Chile es
ese gran patrimonio que posee y que en definitiva es bastante difícil
que tenga alguna baja de un día a otro, esa estabilidad motiva a
invertir. (Quinto, 2023)

Otro factor que motiva es esa baja tasa de reclamos que tiene el banco,
sin ser el mejor, pero al ser baja, siendo uno de los más grandes del
país permite y le da esa confianza al inversionista a ver el banco de
una manera que le da seguridad que tenga pocos reclamos.

También esta el factor de rentabilidad y dividendos del banco, que al
Banco de Chile le da ese status de que, al ser un banco grande, con
prestigio, por lo que genera esa confianza a invertir.

Se eligió este sector sobre el de inmobiliaria y energía, dado que es
una industria que esta muy desarrollada y es algo que si o si tendré que
ver en mi vida profesional, el entender bien como funciona la industria
bancaria, además algo del tema de inversiones vi en mi práctica y me
motiva a profundizar.

\hypertarget{contexto}{%
\subsubsection{Contexto}\label{contexto}}

Con respecto a los activos o variables que posee el Banco de Chile, se
puede decir que son varios, dado el alto patrimonio y posicionamiento
del Banco en el contexto nacional, pero se pueden destacar los
siguientes:

Depósito a Plazo: como dice la CMF, son sumas de dinero que administran
las instituciones financieras, que buscan generar intereses en un
periodo de tiempo. En el Banco de Chile es un activo muy utilizado, dado
que tiene una de las 5 mejores tasas de los Bancos nacionales. Se pueden
ver los siguientes tipos de depósito: (\emph{CMF Educa}, n.d.)

\begin{itemize}
\item
  A largo plazo fijo: en donde se obliga a pagar en un día prefijado. En
  el caso del Banco de Chile, en un deposito de 90 días, tiene una tasa
  de 0,9\% mensual y una tasa promedio de 2,67\%.
\item
  A largo plazo renovable: similar al anterior, pero con opción de
  prórroga. En el caso del Banco de Chile, en un depósito de 90 días,
  tiene una tasa de 0,9\% mensual y una tasa promedio de 1,83\%.
\item
  A plazo indefinido: no tiene fecha de vencimiento.
\end{itemize}

Otro activo o variable corresponde a los prestamos o créditos, en donde
se pueden observar las hipotecas, prestamos de cualquier tipo, tipos de
financiamiento, tarjetas de crédito, entre otras. Y el Banco de Chile es
uno de los más grandes en esta área, dado que tiene el 25\% del mercado
como se mencionó con anterioridad, lo que son 1,9 millones de personas
que optan por servicios como este en el banco.

Otro puede ser los activos como tal del banco, los activos líquidos como
el dinero y le dan liquidez al Banco o activos fijos en los que incluyen
propiedades o los equipos de las sucursales y oficinas, edificios, etc.
Y justamente el Banco de Chile esta en todas las regiones, con más de
400 sucursales de este.

\hypertarget{anuxe1lisis-de-largo-plazo}{%
\subsubsection{Análisis de Largo
Plazo}\label{anuxe1lisis-de-largo-plazo}}

\hypertarget{caracterizaciuxf3n-deuda-aplicaciuxf3n-cap.-15}{%
\paragraph{Caracterización deuda (Aplicación cap.
15)}\label{caracterizaciuxf3n-deuda-aplicaciuxf3n-cap.-15}}

El Banco de Chile cuenta con una gran cantidad de deuda, entre que se
destacan los siguientes:

Bonos: en el caso del Banco de Chile se emiten una gran cantidad de
Bonos al año, el banco ocupa este instrumento de deuda en Unidades de
Fomento, el año pasado se emitieron una gran cantidad de bonos, la
mayoría de largo plazo, con un monto máximo de UF 160.000.000 y fueron
en series. Dada esa gran cantidad, se puede ver la cantidad de deuda que
se obtiene con este tipo de activo. (\emph{Banco De Chile}, 2022)

Acciones nominativas: el Banco de Chile posee este tipo de acciones, las
cuales se emiten a nombre de un titular, llevan el registro de los
accionistas. Según su reporte anual del año 2022, el Banco se compone de
101.017.081.114 acciones nominativas, que se encuentran suscritas y
pagadas, sin valor nominal. Se transan en la bolsa de Santiago.
(\emph{Banco De Chile}, 2022)

Acciones fuera de Chile: también es importante recalcar que desde el 2
de enero del 2002 las acciones del Banco de Chile se transan en la Bolsa
de Valores de Nueva York en el programa de American Depositary Receips
(ADR). Donde cada ADR representan 200 acciones comunes. (\emph{Banco De
Chile}, 2022)

Describir el activo especificando en que realiza sus inversiones (por
tipo de instrumento). Especificar sus acciones comunes, preferentes y
bonos relacionados aplicando lo aprendido en el capitulo respectivo.

Ejemplo de un fondo mutuo como activo a analizar:

\begin{figure}

{\centering \includegraphics{photos/Captura de pantalla 2023-08-22 112340.png}

}

\end{figure}

\hypertarget{caracterizaciuxf3n-de-emisiones-histuxf3ricas-aplicaciuxf3n-cap.-20}{%
\paragraph{Caracterización de emisiones históricas (Aplicación cap.
20)}\label{caracterizaciuxf3n-de-emisiones-histuxf3ricas-aplicaciuxf3n-cap.-20}}

Sobre la emisión de acciones como tal se destaca lo dicho anteriormente,
estas 101.017.081.114 acciones nominativas, las cuales van en aumento,
dado que cada año se van emitiendo más acciones del banco,

Sobre la emisión de bonos, se destacan varios, como se menciono en la
parte de caracterización de la deuda, sin embargo, entrando en
actualidad se puede destacar uno que se emitió en México, en donde se
emitió un bono por US\$40 millones para financiar proyectos liderados
por mujeres (\emph{Banco De Chile Emite Bono Por US\$ 40 Millones Para
Financiar Emprendimientos Liderados Por Mujeres}, 2023) por poner algún
ejemplo más actual.

Describir el proceso de emision de acciones paso a paso del activo
seleccionado, caracterizando el tipo de colocacion que utilizaron en
contraste con el procedimiento basico realizado en el capitulo
respectivo.

\begin{figure}

{\centering \includegraphics{photos/Captura de pantalla 2023-08-22 111527.png}

}

\end{figure}

\hypertarget{relaciuxf3n-con-activos-derivados}{%
\paragraph{Relación con activos
derivados}\label{relaciuxf3n-con-activos-derivados}}

Dado que en el contexto nacional no existen muchos derivados, la
situación del Banco de Chile es parecida, pero como el banco posee
inversiones en el extranjero, como se menciono en la sección de
caracterización de deuda, el Banco tiene acciones fuera, por lo que son
activos derivados.

Respecto a la evolución del precio de los derivados se pueden ver los
siguientes datos, obtenidos del reporte anual del Banco de Chile, en
realidad es un resumen de los datos, dado que no se logro copiar y pegar
la tabla completa (\emph{BANCO DE CHILE Y SUS FILIALES}, 2022):

Tabla 1: Monto Nacional de contratos con vencimiento final

\begin{longtable}[]{@{}
  >{\raggedright\arraybackslash}p{(\columnwidth - 4\tabcolsep) * \real{0.3514}}
  >{\raggedright\arraybackslash}p{(\columnwidth - 4\tabcolsep) * \real{0.3243}}
  >{\raggedright\arraybackslash}p{(\columnwidth - 4\tabcolsep) * \real{0.3243}}@{}}
\toprule\noalign{}
\endhead
\bottomrule\noalign{}
\endlastfoot
& Valor razonable activo & Valor razonable activo \\
& Septiembre & Diciembre \\
& 2022 & 2021 \\
& MMS & MMS \\
Forwards de monedas & 903.092 & 742.545 \\
Swap de tasas de interes & 1.810.793 & 825.525 \\
Swap de monedas y tasas & 1.434.776 & 1.132.718 \\
Opciones Call Moneda & 9.341 & 4.509 \\
Opciones Pull Moneda & 471 & 199 \\
Total & 4.158.473 & 2.705.496 \\
\end{longtable}

\hypertarget{reporte-grupal}{%
\subsubsection{Reporte grupal}\label{reporte-grupal}}

Definicion de un balance de portafolio sobre los 3 activos del grupo,
donde se especifique los pesos de cada activo de tal forma que maximize
el retorno de la cartera.

\hypertarget{referencias}{%
\subsubsection{Referencias}\label{referencias}}

Banco de Chile. (2022, November 1). Banco de Chile. Retrieved September
28, 2023, from
https://portales.bancochile.cl/uploads/000/040/621/268c71db-b47a-4977-b69b-def5308d5bf8/original/Prospecto\_Despues\_Certificado\_CMF.pdf

Banco de Chile emite bono por US\$ 40 millones para financiar
emprendimientos liderados por mujeres. (2023, June 1). Banco de Chile.
Retrieved September 28, 2023, from
https://portales.bancochile.cl/nuestrobanco/novedades/emprendimiento/detalles/banco-de-chile-emite-bono-por-us-40-millones-para-financiar-emprendimientos-liderados-por-mujeres

BANCO DE CHILE Y SUS FILIALES. (2022, September 30). Banco de Chile.
Retrieved September 28, 2023, from
https://portales.bancochile.cl/uploads/000/040/296/a1d50323-f342-4c97-b11b-0890284d2594/original/EEFF\_Banco\_de\_Chile\_y\_filiales\_09-2022.pdf

CMF Educa. (n.d.). CMFEduca.cl. Retrieved September 28, 2023, from
https://www.cmfeduca.cl/educa/621/w3-channel.html

Cohn, A., Fehr, E., \& Maréchal, M. A. (2014). Business culture and
dishonesty in the banking industry. Nature, 516(516), 86-89.
https://www.nature.com/articles/nature13977

¿Cuáles son los mejores depósitos a plazo para 2023? (2023, September
26). Rankia Chile. Retrieved September 28, 2023, from
https://www.rankia.cl/blog/mejores-depositos-a-plazo/3257869-cuales-son-mejores-depositos-plazo

Quinto, C. (2023, September 12). Mejores bancos de Chile 2023 - Rankia.
Rankia Chile. Retrieved September 28, 2023, from
https://www.rankia.cl/blog/mejores-depositos-a-plazo/3097703-mejores-bancos-chile

SANTIAGOX. (n.d.). SANTIAGOX. Retrieved September 28, 2023, from
https://www.bolsadesantiago.com/resumen\_instrumento/CHILE



\end{document}
